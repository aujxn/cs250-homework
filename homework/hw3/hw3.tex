\documentclass{exam}
\usepackage{amsmath}
\usepackage{xlop} % for addition and multiplication
\usepackage{array} % for right columns in tabular
\usepackage{url}

% code for making a fixed width right aligned column in tabular
\newcolumntype{x}[1]{%
>{\raggedleft\hspace{0pt}}p{#1}}%


\newcommand\N{\mathbf{N}}
\newcommand\Z{\mathbf{Z}}
\newcommand\E{\mathbf{E}}
\renewcommand\O{\mathbf{O}}
\newcommand\Q{\mathbf{Q}}
\renewcommand\P{\mathbf{P}}
\def\land{\wedge}
\def\lor{\vee}
\def\xor{\oplus}
\def\T{\top}
\def\F{\bot}
\def\lnot{\neg}

\usepackage{tikz}
\usepackage{tikz-3dplot}
\usetikzlibrary{calc,shapes.multipart,chains,arrows,matrix}
\usetikzlibrary{graphs}
\usetikzlibrary{graphs.standard}
\usetikzlibrary{graphdrawing}
\usegdlibrary{force}
\usegdlibrary{layered}

\title{CS250 homeowrk 3}
\author{name:Austen Nelson}
\date{Due: 7/16/19}

\begin{document}
\maketitle

\begin{questions}

\question
Prove that $n^2\%3 \ne 2$ for all integers $n$.\\
Remember $n\%3$ is the remainder of $n/3$, so it can only ever be 0, 1, or 2. \\\\
Proof by cases: \\
$x \in \N \colon 3 \mid x$ and $x = 3k \colon k \in \N$  because of definition of divides \\
   \begin{enumerate}
      \item case 1: ${x^2 \equiv 0}$ (mod 3)
   \begin{align*}
      x^2 &= (3k)^2 &&\\
         &= 9k^2 &&\\
         &= 3(3k^2) &&\\
      {3(3k^2) \equiv 0} \text{ (mod 3)} && \text{case 1 is true because of definition of modulus}
   \end{align*}
\item case 2: ${(x+1)^2 \equiv 1}$ (mod 3)
      \begin{align*}
         (x+1)^2 &= x^2+2x+1 \\
            &= 9k^2 +6k + 1 \\
            &= 3(3k^2+2k) + 1 \\
            3(3k^2+2k) \mod 3 = 0 &&& \text{definition of modulus} \\
            {1 \equiv 1} \text{ (mod 3)} &&& \text{case 2 is true through substitution}
      \end{align*}
   \item case 3: ${(x+2)^2 \equiv 1}$ (mod 3)
      \begin{align*}
         (x+2)^2 &= x^2 +4x + 4 &&\\
            &=9k^2 + 12k + 4 &&\\
            &=3(3k^2 +4k) + 4 &&\\
            3(3k^2 +4k) \mod 3 = 0 &&& \text{definition of modulus} \\
            {4 \equiv 1} \text{ (mod 3)} &&& \text{case 3 is true through substitution}
      \end{align*}
   \end{enumerate}
   These are all the cases we need to prove that $x^2 \mod 3 \in \{0, 1\}$. To demonstrate this:
   \begin{align*}
      (x+3)^2 = x^2 + 6x + 9 \\
      6x + 9 = 3(2x + 3) \\
      3(2x+3) \mod 3 = 0 \\
      {(x+3)^2 \equiv x^2} \text{ (mod 3)}
   \end{align*}
\vspace{2in}

\question
Fermat's Last theorem is a famous theorem in Math that was unproven for 200 years.
The theorem says for all $n > 2, a,b,c\in \N. a^n + b^n \ne c^n$.
Or $a^n + b^n = c^n$ has no integer solutions for $n$ larger than 2.
Use this theorem to prove that $\sqrt[n]{2}$ is irrational for $n$ larger than 2.
\\\\
   \begin{align*}
      &\forall n > 2,a,b,c \in \N. \\
      &a^n+b^n \ne c^n \\
      &\text{assume } \sqrt[n]{2} = \frac{p}{q} \colon p,q \in \N \\
      &2 = \frac{p^n}{q^n} \\
      &2q^n = p^n \\
      &q^n + q^n = p^n \text{ which is a contradiction with Fermat's. So our assumption must be false.} \\
   \end{align*}

\question
\begin{parts}
\part Prove that there is no smallest positive rational number greater than 0.\\
      Assume x is the smallest positive rational number greater than 0. \\
   \begin{align*}
      &x \in \Q^+, p \in \N, q \in \N. && \\
      &x = \frac{p}{q} && \text{definition of a rational number} \\
      &\frac{p}{q} > \frac{p}{q+1} && \text{by incrementing q we create a smaller rational than x, which is a contradiction}
   \end{align*}

\part Prove that for every positive real number greater than 0 there is a smaller positive rational number.\\
(Hint: if $r < 1$ then $1/r > 1$)\\
Let x be a real number between 0 and 1. \\
Let y be the ceiling of 1/x. \\
1/y > x. \\
\part Now Prove that there is no smallest positive real number greater than 0.\\
Let x be the smallest positive real number greater than 0. \\
Let y be x/2. This is a contradiction because y < x. There must be no smallest positive real number. \\
\end{parts}

\question
In a graph $G$ we have a relation $u \sim v$ if $u$ and $v$ have an edge between them.\\
Is this relation reflexive, symmetric, antisymmetric, transitive. \\\\
This relation is not reflexive because a vertix is not required to have an edge to itself. For an undirected graph, it is symmetric (but not for a directed graph). An undirected graph cannot be antisymmetric (some directed graphs could possibly be antisymmetric). It is not transitive. If vertex A has an edge to B and B to C, there is a path from A to C (not an edge). A complete graph would be transitive.\\
\question
remember a graph is a bunch of vertices connected by edges.\\
A \textit{path} is a sequence of vertices $v_1,v_2,\ldots$ where there is an edge between every vertex.\\
A \textit{cycle} is a path that starts and ends at the same vertex.\\
Prove that if a graph has no cycles, then there is at most one path between any two vertices. \\\\
Suppose there are two different paths between nodes a and b in the graph. Trace the paths simultaneously. When the paths separate and join again they create a cycle.

\question
The degree of a vertex in a graph is the number of vertices it's connected to.
so $deg(v) = |\{u : u \sim v\}|$
For the following graph give the degree of each vertex.\\

\tikz \graph [spring layout, nodes={draw, circle, inner sep=0pt, fill=black}] {
    a -- o -- c ;
    c -- a ;
    o -- s -- e -- c ;
};\\

\question
Prove that for the vertices $\{v_1,v_2,\ldots v_n\}$ in a graph that
$deg(v_1) + deg(v_2) + \ldots deg(v_n) = 2\cdot |E|$.\\
Every edge added to a graph has to connect two vertices. This means that the number of vertices will always be 2 times the total number of edges.
\end{questions}
\end{document}
