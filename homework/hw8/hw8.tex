
\documentclass{article}

\usepackage{fancyhdr}
\usepackage{empheq}
\usepackage{extramarks}
\usepackage{amsmath}
\usepackage{amssymb}
\usepackage{amsthm}
\usepackage{amsfonts}
\usepackage{tikz}
\usepackage[plain]{algorithm}
\usepackage{algpseudocode}
\usepackage{enumerate}
\usepackage{url}

\usetikzlibrary{automata,positioning}

%
% Basic Document Settings
%

\topmargin=-0.45in
\evensidemargin=0in
\oddsidemargin=0in
\textwidth=6.5in
\textheight=9.0in
\headsep=0.25in

\linespread{1.1}

\pagestyle{fancy}
\lhead{\hmwkAuthorName}
\chead{\hmwkClass\ (\hmwkClassInstructor\ \hmwkClassTime): \hmwkTitle}
\rhead{\firstxmark}
\lfoot{\lastxmark}
\cfoot{\thepage}

\renewcommand\headrulewidth{0.4pt}
\renewcommand\footrulewidth{0.4pt}

\setlength\parindent{0pt}

\newcommand\dplus{\ensuremath{\mathbin{+\mkern-10mu+}}}
\newcommand*\widefbox[1]{\fbox{\hspace{.5em}#1\hspace{.5em}}}

%
% Create Problem Sections
%

\newcommand{\enterProblemHeader}[1]{
   \nobreak\extramarks{}{Problem \arabic{#1} continued on next page\ldots}\nobreak{}
   \nobreak\extramarks{Problem \arabic{#1} (continued)}{Problem \arabic{#1} continued on next page\ldots}\nobreak{}
}

\newcommand{\exitProblemHeader}[1]{
   \nobreak\extramarks{Problem \arabic{#1} (continued)}{Problem \arabic{#1} continued on next page\ldots}\nobreak{}
   \stepcounter{#1}
   \nobreak\extramarks{Problem \arabic{#1}}{}\nobreak{}
}

\setcounter{secnumdepth}{0}
\newcounter{partCounter}
\newcounter{homeworkProblemCounter}
\setcounter{homeworkProblemCounter}{1}
\nobreak\extramarks{Problem \arabic{homeworkProblemCounter}}{}\nobreak{}

%
% Homework Problem Environment
%
% This environment takes an optional argument. When given, it will adjust the
% problem counter. This is useful for when the problems given for your
% assignment aren't sequential. See the last 3 problems of this template for an
% example.
%
\newenvironment{homeworkProblem}[1][-1]{
   \ifnum#1>0
   \setcounter{homeworkProblemCounter}{#1}
   \fi
   \section{Problem \arabic{homeworkProblemCounter}}
   \setcounter{partCounter}{1}
   \enterProblemHeader{homeworkProblemCounter}
}{
   \exitProblemHeader{homeworkProblemCounter}
}

%
% Homework Details
%   - Title
%   - Due date
%   - Class
%   - Section/Time
%   - Instructor
%   - Author
%

\newcommand{\hmwkTitle}{Homework\ \#8}
\newcommand{\hmwkDueDate}{August 27, 2019}
\newcommand{\hmwkClass}{CS250}
\newcommand{\hmwkClassTime}{Summer19}
\newcommand{\hmwkClassInstructor}{Steven Libby}
\newcommand{\hmwkAuthorName}{\textbf{Austen Nelson}}

%
% Title Page
%

\title{
   \vspace{2in}
   \textmd{\textbf{\hmwkClass:\ \hmwkTitle}}\\
   \normalsize\vspace{0.1in}\small{Due\ on\ \hmwkDueDate\ at 2:15pm}\\
   \vspace{0.1in}\large{\textit{\hmwkClassInstructor\ \hmwkClassTime}}
   \vspace{3in}
}

\author{\hmwkAuthorName}
\date{}

\renewcommand{\part}[1]{\textbf{\large Part \Alph{partCounter}}\stepcounter{partCounter}\\}

%
% Various Helper Commands
%

% Useful for algorithms
\newcommand{\alg}[1]{\textsc{\bfseries \footnotesize #1}}

% For derivatives
\newcommand{\deriv}[1]{\frac{\mathrm{d}}{\mathrm{d}x} (#1)}

% For partial derivatives
\newcommand{\pderiv}[2]{\frac{\partial}{\partial #1} (#2)}

% Integral dx
\newcommand{\dx}{\mathrm{d}x}

% Alias for the Solution section header
\newcommand{\solution}{\textbf{\large Solution}}

% Probability commands: Expectation, Variance, Covariance, Bias
\newcommand{\E}{\mathrm{E}}
\newcommand{\Var}{\mathrm{Var}}
\newcommand{\Cov}{\mathrm{Cov}}
\newcommand{\Bias}{\mathrm{Bias}}

\begin{document}

% \maketitle

% \pagebreak

\begin{homeworkProblem}
Give The $\Theta$ for the following.\\

Justify your answer.

\[
   f(n)  \in \Theta (g(n)) \iff n_0 \in \mathbf{N}, \;
   \forall n > n_0, \; \exists c_1, c_2 \in \mathbf{R} \colon
   c_1 \cdot g(n) \leq f(n) \leq c_2 \cdot g(n)
\]

   \begin{flalign}
      f(n) & \in \Theta (g(n)) && c_1 && c_2 &\\
      5x^2 + 4x + 3 & \in \Theta (x^2) && 1 && 12 &\\
      2^n + n! & \in \Theta (n!) && 1 && 3 &\\
      n^2 + 2^n & \in \Theta (2^n) && 1 && 3 &\\
      \log{n} + n & \in \Theta (n) && 1 && 2&
   \end{flalign}

   The last one is the only one that cannot be verified by setting up a simple inequality. \\
   Upper Bound:
   \begin{flalign*}
      \log{n!} &= \log{1} + \log{2} + \log{3} + \cdots + \log{n} &\\
               &= \sum_{i=1}^{n} \log{i} &\\
               &\leq n\log{n} && \text{There are $n\log{x}$ and each $\log{x}$ is $\leq \log{n}$}&\\
               c_2 &= 1&
   \end{flalign*}

   Lower Bound:
   \begin{flalign*}
      &= \sum_{i=1}^{n} \log{i} &\\
      &\geq \sum_{i=\frac{n}{2}}^{n} \log{i} && \text{chop off the first half of the sum}&\\
      &\geq \sum_{i=\frac{n}{2}}^{n} \log{\frac{n}{2}} && \text{$\frac{n}{2} \leq i$ always}&\\
      &\geq \frac{n}{2} \log{\frac{n}{2}} && \text{reduce sum} &\\
  c_1 &= \frac{1}{2}&
   \end{flalign*}
\end{homeworkProblem}
\pagebreak

\begin{homeworkProblem}
Give a closed form for the following, then give the $\Theta$ \\

Master Theorem:

\begin{flalign*}
   T(n) &= aT\left(\frac{n}{b}\right) + f(n) &\\
        &= a^k \cdot T(1) + \sum_{i=0}^{k-1}a^i \cdot f\left(\frac{n}{b^i}\right) &\\
   T(n) &\in 
      \begin{cases}
         \Theta(n^{\log_b{a}})               &f(n) \in \mathbf{O}(n^{\log_b{a}})\\
         \Theta(n^{\log_b{a}}\cdot \log n)   &f(n) \in \Theta(n^{\log_b{a}})\\
         \Theta(f(n))                        &f(n) \in \Omega(n^{\log_b{a}})
      \end{cases}&
\end{flalign*}

\begin{enumerate}[{(a)}]
\item
\begin{flalign*}
   a_0 &= 5&\\
   a_n &= 3a_{n-1}&\\
       &= 3(3a_{n-2})&\\
       &= 3(3(3a_{n-3}))&\\
       &= 3^n(a_0)&\\
       &= 3^n \cdot 5 \in \Theta (3^n)&
\end{flalign*}
\item
\begin{flalign*}
   a_4 &= 2&\\
   a_n &= a_{n-1} + \log_2(n)&\\
       &= a_{n-2} + \log_2(n-1) + \log_2(n)&\\
       &= a_{n-3} + \log_2(n-2) + \log_2(n-1) + \log_2(n)&\\
       &= a_4 + \log_2\left(\frac{n!}{4!}\right)&\\
       &= 2 + \log_2 \frac{n!}{24} \in \Theta (n\log{n})&
\end{flalign*}
\item
\begin{flalign*}
   a_1 &= 1&\\
   a_n &= 2a_{n-2} + 1&\\
       &= 2(2a_{n-4} + 1) + 1&\\
       &= 2(2(2a_{n-6} +1) + 1) + 1&\\
       &= 2^{\frac{n-1}{2}} + \sum_{i=1}^{\frac{n-1}{2}} 2^{i-1}&\\
       &= a_1 \cdot 2^{\frac{n+1}{2}}&\\
       &=2^{\frac{n+1}{2}} \in \Theta(2^n)&
\end{flalign*}
\item
\begin{flalign*}
   T(1) &=1&\\
   T(n) &=3T\left(\frac{n}{2}\right) + 1&\\
   a    &=3&\\
   b    &=2&\\
   k    &=\log_2{n}&\\
   f(n) &=1&\\
   T(n) &=3^{\log_2{n}} + \sum_{i=1}^{\log_2{n}-1} 3^i&\\
        &=n^{\log_2{3}} + \frac{3^{\log_2{n}} - 1}{2} &\\
        &=\frac{3n^{\log_2{3}} - 1}{2} \in \Theta(n^{\log_2{3}})&
\end{flalign*}
\item
\begin{flalign*}
   T(1) &=4&\\
   T(n) &=T\left(\frac{k}{3}\right) + 4&\\
   a    &=1&\\
   b    &=3&\\
   k    &=\log_3{n}&\\
   f(n) &=4&\\
   T(n) &=1^{k} 4 + \sum_{i=1}^{\log_{3}{n}-1}1^i 4&\\
        &= 4 + 4(\log_{3}{n}-1) \in \Theta(\log n)&
\end{flalign*}
\item
\begin{flalign*}
   T(n) &=3T(n-2) + 4(n-2) + 2&\\
        &=3(3T(n-4) + 4(n-4) + 2) + 4(n-2) + 2&\\
        &=3(3(3T(n-6) + 4(n-6) + 2) + 4(n-4) + 2) + 4(n-2) + 2&\\
        &=3^{\frac{n}{2}} + \sum_{i=1}^{\frac{n}{2}}3^{\frac{n}{2}-i} \cdot (4(2i-2) + 2)&\\
        &=3^{\frac{n}{2} + 1} - 2n + 3^{\frac{n}{2}}-3 \in \Theta(3^n)&
\end{flalign*}
\end{enumerate}
\end{homeworkProblem}
\pagebreak

\begin{homeworkProblem}
   Prove \textbf{theorem 2:} $x^k \in \mathbf{O}(x^{k + c})$ \\
   \\
   Let $c_1 = 1$ and assuming $k \in \mathbf{N}$ \\
   If $x^k \leq c_1 \cdot x^{k+c}$ then $x^k \in \mathbf{O}(x^{k + c})$ \\
   $\frac{x^k}{x^{k+c}} \leq c_1$ \\
   $\frac{1}{x^c} \leq c_1$ is true for all positive c and $x>k$.
\end{homeworkProblem}

\begin{homeworkProblem}
   Prove \textbf{theorem 3:} $x^k + c\cdot x^{k-r} \in \mathbf{O}(x^k)$ \\
   \\
   Assuming $k,r,x \in \mathbf{N}\\$
   Let $c_1 = 1 + c$\\
   If $x^k + c\cdot x^{k-r} \leq c_1(x^k)$ for some $c_1$ then $x^k + c\cdot x^{k-r} \in \mathbf{O}(x^k)$ \\
   $1 + c\cdot x^{-r} \leq c_1$ \\
   $\frac{1}{x^r} \leq 1$ will always be true.

\end{homeworkProblem}

\begin{homeworkProblem}
   Prove \textbf{theorem 5:} if $f(n) \in \mathbf{O}(g(n))$ and $g(n) \in \mathbf{O}(h(n))$, then $f(n) \in \mathbf{O}(h(n))$\\
   \\
   If $f(n) \in \mathbf{O}(g(n))$ then there exists a $c_1$ such that $f(n) \leq c_1 \cdot g(n)$\\
   and if $g(n) \in \mathbf{O}(h(n))$ then there exists a $c_2$ such that $g(n) \leq c_2 \cdot h(n)$\\
   Let $c_3 = c_1 \cdot c_2$. Then $f(n) \leq c_1 g(n) \leq c_3 h(n)$\\
   Due to the transitivity of $\leq$, it must be that $f(n) \in \mathbf{O}(h(n))$

\end{homeworkProblem}

\begin{homeworkProblem}
Give the $\Theta$ running time for the following selection sort algorithm
\begin{verbatim}
def selSort(l):
    for i in range(len(l)):
        min = l[i]
        minI = i
        for j in range(i,len(l)):
            if l[j] < min:
                minI = j
                min = l[j]
            #end if
        # end for
        (l[i], min) = (min, l[i])
    # end for
\end{verbatim}
For the first element you have to compare it with $n-1$ items.

For the second element you have to compare with $n-2$ items.

The total number of compares is $\sum_{i=1}^{n} n-i$ or $\frac{n^2-n}{2} \in \Theta(n^2)$

\end{homeworkProblem}

\begin{homeworkProblem}
\begin{verbatim}
def badSort(l):
    n = len(l)

    if n == 1:
        return l

    first = badSort(l[0:n-2])
    middle = badSort(l[1:n-1])
    end = badSort(l[2:n])

    return [first[0]] + middle + [end[n-1]]
\end{verbatim}
   \begin{enumerate}[{(a)}]
      \item Give the recurrence relation for badSort\\
         remember \texttt{l[a:b]} copies the elements from \texttt{l[a]} to \texttt{l[b]}\\
         \\
         Assuming no copies are made on the return (I'm sure there are but
         I have no idea how the python list is implemented)
         each recursive call only takes 2 items off the length and 3 calls are made.\\
            $T(n) = 3T(n-2)$ or $T(n) = 3^{n-1}{2}$ \\
    \item Give the $\Theta$ for badSort \\
       \\
       $T(n) \in \Theta(3^n)$
\end{enumerate}

\end{homeworkProblem}

\begin{homeworkProblem}
The following algorithm is the merge sort we way in class
\begin{verbatim}
def merge(low, high):
    i = 0
    j = 0
    merged = []
    while i < len(low) and j < len(high):
        if low[i] < high[j]:
            merged += [low[i]]
            i += 1
        else:
            merged += [high[j]]
            j += 1
    return merged + low[i:] + high[j:]

def mergeSort(lst):
    n = len(lst)
    n2 = int(n/2)

    # base case, if our list is 0, or 1 element, the its sorted
    if n <= 1:
        return lst

    # recursive case: split the list in half
    # sort the halves
    # merge the lists together
    low = mergeSort(lst[0:n2])
    high = mergeSort(lst[n2:n])
    lst = merge(low,high)

    return lst
\end{verbatim}
   \begin{enumerate}[{(a)}]
      \item Give the $\Theta$ running time for merge.\\
          hint: what is the input size for merge? \\
          \\
          Merge does one compare and one move for each of the items in the two inputs. \\
          This means merge is linear time $\in \Theta(n)$
    \item Use part 1 to give a recurrence relation for the running time of mergeSort\\
       \\
         $T(n) = 2T(\frac{n}{2}) + n$
    \item Solve the recurrence to get a $\Theta$ running time for mergesort.

       \begin{flalign*}
          a &= 2 &\\
          b &= 2 &\\
          f(n) &= n &\\
          k &=\log_{2}{n} &\\
          T(n) &=2^{\log_2{n}} \cdot 1 + \sum_{i=0}^{\log_2{n} - 1} 2^i \cdot \frac{n}{2^i} &\\
               &=n + n \cdot \log_2{n} &\\
          &\text{master theorem:} &\\
          f(n) &\in \Theta (n^{\log_{2}{2}}) &\\
          T(n) &\in \Theta (n^{\log_{2}{2}} \cdot \log n) &\\
          T(n) &\in \Theta (n\log{n})&
       \end{flalign*}

   \end{enumerate}
\end{homeworkProblem}
\end{document}
