\documentclass{exam}
\usepackage{amsmath}
\usepackage{xlop} % for addition and multiplication
\usepackage{array} % for right columns in tabular
\usepackage{url}
\usepackage[procnames]{listings}
\usepackage{color}

% code for making a fixed width right aligned column in tabular
\newcolumntype{x}[1]{%
>{\raggedleft\hspace{0pt}}p{#1}}%


\newcommand\N{\mathbf{N}}
\newcommand\Z{\mathbf{Z}}
\newcommand\E{\mathbf{E}}
\renewcommand\O{\mathbf{O}}
\newcommand\Q{\mathbf{Q}}
\renewcommand\P{\mathbf{P}}
\def\land{\wedge}
\def\lor{\vee}
\def\xor{\oplus}
\def\T{\top}
\def\F{\bot}
\def\lnot{\neg}

\lhead{Austen Nelson}
\rhead{HW4}

\title{CS250 homework 4}
\author{name:\underline{Austen Nelson}}
\date{Due: 7/23/19}

\begin{document}
\maketitle

\begin{questions}

\definecolor{keywords}{RGB}{255,0,90}
\definecolor{comments}{RGB}{0,0,113}
\definecolor{red}{RGB}{160,0,0}
\definecolor{green}{RGB}{0,150,0}
 
\lstset{language=Python, 
        basicstyle=\ttfamily\small, 
        keywordstyle=\color{keywords},
        commentstyle=\color{comments},
        stringstyle=\color{red},
        showstringspaces=false,
        identifierstyle=\color{green},
        procnamekeys={def,class}}
 
\question
The Fibonacci numbers are a fun sequence of numbers that show up in math a lot.
The sequence is $1,1,2,3,5,8,13,21,\ldots$, and we compute each number by adding the previous two.
Write an iterative (while loop) python function to compute the $n^{th}$ Fibonacci number.
so \texttt{fib(6)} should return \texttt 8.
\begin{lstlisting}
def fib_iter(a):
   num0 = 0
   num = 1

   for i in range(a - 1):
      old = num
      num = num + num0
      num0 = old

   return num
\end{lstlisting}

\question
Now write a recursive program to compute the $n^{th}$ Fibonacci number.
\begin{lstlisting}
def fib_rec(a):
   if a <= 1:
      return a

   return fib_rec(a - 1) + fib_rec(a - 2)
\end{lstlisting}

\question
Let $e = gcd(n,m\%n)$, Prove that $e | m$ and $e | n$ \\
\begin{align*}
   n, m, e, x, y, r, q, z \in \N && \text{variable declarations} \\
   m \mod n = r && \text{initializing r} \\
   m = qn + r && \text{rewriting modulus} \\
   e \mid n && \text{definition of gcd} \\
   n = ze \text{ so } qn \equiv qze && \text{substitution} \\
   e \mid qn \text{ or } qn = xe && \text{because $qn = qze$ and definition of divides} \\
   e \mid r \text{ or } r = ye && \text{definition of gcd and definition of divides} \\
   m = xe + ye \text{ or } m = e(x+y) && \text{substitution and algebra} \\
   e \mid m && \text{definition of divides}
\end{align*}

\question
Prove that for any $a,b,c\in \N$ if $a | b$ and $a | c$ then $a | bx + cy$ for any $x,y\in \N$.
\begin{align*}
   a, b, c, x, y, s, z \in \N && \text{variable declarations} \\
   b = az \text{ and } c = as && \text{definition of divides} \\
   bx + cy \equiv azx + asy \equiv a(zx + sy) && \text{substitution and factor} \\
   a \mid a(zx + sy) && \text{definition of divides} \\
\end{align*}

\question
The distance between two vertices in a graph is the length of the shortest path between them.\\
We usually write the distance between $u$ and $v$ in graph $G$ as $d(u,v)$.\\
Show that for any three vertices $u,v,t \in G$ $d(u,v) \le d(u,t) + d(t,v)$. \\
\\
Case 1: t is on the shortest path from u to v. $d(u,t) + d(t,v) = d(u,v)$ because if there exists a shorter path (u,t) or (t,v) then there exists a path shorter than our shortest path (u,v).
\\
Case 2: t is not on the shortest path from u to v. Path (u,t) must diverge from shortest path (u,v). $d(u,t) + d(t,v) > d(u,v)$ because path $u \to t \to v$ is not the shortest path (u,v).

\question
In the video \url{https://www.youtube.com/watch?v=2SUvWfNJSsM}
he shows that you can solve towers of hanoi by counting in binary.
Make this explicit by writing a python function \texttt{move\_disk(n)} where \texttt n
is a binary number.
\texttt{move\_disk} should return the number of the disk to be moved.
So, to move the $3^{rd}$ disk, you'd return 3.
You can test your code in \texttt{hanoi.py} on d2l.
Right now it just loops forever

\begin{lstlisting}
def move_disk(i):
   h = 5
   while h > 0:
      if i % 2**h == 0:
         return h + 1
      h = h - 1
   return 1
\end{lstlisting}

\question
give a bijection as a python function for
\begin{itemize}
    \item $\E \to \N$

\begin{lstlisting}
def E_N(even)
   return even/2
\end{lstlisting}

    \item $\N \to \Z$

\begin{lstlisting}
import math

def N_Z(nat)
   return (-1 ** nat) * math.floor(nat / 2)
\end{lstlisting}

    \item $\N \to \Q^+$

\begin{lstlisting}
import math
from fractions import Fraction

def N_Qpos(nat)
   z = 0

   #Calkin-Wilf series
   for i in range(nat):
      z = Fraction(1, 2 * math.floor(z) - z + 1)

   return z
\end{lstlisting}

    \item $\Z \to \Q$

\begin{lstlisting}
def Z_Q(an_int)
   if an_int == 0:
      return 0

   if an_int > 0:
      return N_Qpos(an_int)

   return -1 * N_Qpos(-1 * an_int)
\end{lstlisting}

    \item $\E \to \Q$

\begin{lstlisting}
def E_Q(even)
   return Z_Q(N_Z(E_N(even)))
\end{lstlisting}

\end{itemize}

\question

Let's try to find a use for this infinity nonsense.\\
$\ $\\
In computer science it can be useful to look at problems as a language.\\
A language is just a set of finite strings.\\
So we can make a language describing $\pi$ as $L_\pi = \{"3", "3.1", "3.14", "3.141", \ldots\}$\\
$\ $\\
So why do we care about languages?\\
Well, we can phrase all of our problems in CS as different languages.\\
For example $L_{factor} = \{"6=2\cdot3", "12=2\cdot2\cdot3", \ldots\}$\\
is the language of numbers and their factors.\\
We can make the language of graphs and their shortest paths, the languages of lists and their sorting.
Really we can make a language for any problem.\\
$\ $\\
A language is decidable if there is a program that can (eventually) produce any string in that language.\\
$\ $\\
We want to prove that there is at least 1 undecidable language.\\
That is, there is a problem that can't be solved by a program.\\
$\ $\\
First show that there are countably many programs we can write.\\
Hint: what happens when you compile a program?\\
A program is just a string. A string can be represented as a binary number. Therefore, a program is just a number within $\N$. This means we have an injection from programs to $\N$ and that there are countably infinite possible programs. \\
second show that there are uncountably many languages.\\
As stated in the example every real number can be represented as a language. This means we there exists a surjection from programs to the reals and that there is an uncountable number of languages.

\end{questions}
\end{document}
